\chapter{Lezione 6 - Struttura, geometria e polarità delle molecole}

\section{Teoria del legame covalente (Lewis)}
Secondo la struttura ideata da Lewis è come se gli elettroni a energia più esterna venissero messi in comune fra tutti gli elementi. Solo gli elettroni di valenza, quindi, vengono raffigurati nella simbologia di Lewis, e sono visualizzabili come punti. 

\textbf{Regola dell'ottetto}: gli atomi tendono a chiudere l'ottetto degli elettroni di valenza, tranne H e He, che essendo in orbitale 1s, possono contenere al più 2 elettroni. 

%%ci starebbe inserire la simbologia di lewis, trova pacchetto

Si parla di \emph{doppietta di elettroni}, poichè bisogna ricordare che gli elettroni si muovono sempre in coppia, il legame viene rappresentato con un trattino, i \emph{doppietti solitari} sono quelli che non sono coinvolti in alcun legame. L'ordine di legame è dato da $e^{-}/2$. 

%%aggiungere gli esempi successivamente 

Un \textbf{legame multiplo} è più corto e più forte di un legame semplice. E' importante anche notare come ogni atomo tende ad assumure la configurazione elettronica del gas nobile più vicino, verso destra o verso sinistra, sarà quindi facile che è un metallo perda un elettrone piuttosto che un non metallo, difatti quest'ultimo tenderà ad acquisirne piuttosto che a perderlo


\subsection{Istruzioni per struttura di Lewis}
\begin{enumerate}
    \item Conto gli elettroni di valenza totali e divido in doppietti, chiaramente se ho una carica (ione) devo tenerne conto, sommando o sottraendo ulteriori elettroni. Solitamente si tratta di numero di elettroni pari altrimenti si parla di \textbf{radicali}, ovvero specie con elettrone spaiato. 
    \item Costruisco lo scheletro della molecola con legami semplici. \textbf{Cos'è la valenza?} E' il numero di legami covalenti formati da un atomo hanno valenza fissa. Un atomo legato solo ad un altro è detto \emph{periferico} mentre se è legato a più atomi è detto \emph{centrale}. 
    In generale gli atomi più elettronegativi sono quelli periferici e quelli meno elettronegativi, se non monovalenti, sono centrali. 
    %%diseni
    \item Assegno ad ogni legame uno dei doppietti precedentementi calcolati, quindi sono le coppie di legame. 
    \item Distribuisco i doppietti rimanenti agli atomi periferici come coppie solitarie (CS). 
    \item Distribuisco i doppietti rimanenti sull'atomo centrale come CS rispettando la regola degli orbitali (CS+CL non può accedere a $n^2$), questo ovviamente è verificabile dal fatto che dal terzo periodo ho più elettroni di valenza, quindi non mi limito più ad 8. 
    \item Riarrangio i doppietti per riportare la carica formale (CF) di ciascun atomo il più possibile vicino a 0, rispettando la regola degli orbitali. \[CF=\text{$e^{-}$ valenza} - CS - \frac{CL}{2}\]
    %%slide 13
\end{enumerate}

Seguendo questa procedura si arriva ad una struttura di Lewis, almeno soddisfacente. 

\subsubsection{Risonanza}
Possibilità degli elettroni di formare sia legami doppi che singoli. 

%%disegno
La molecola è un \textbf{ibrido di risonanza} delle 3 strutture limite e ciò si indica con la simbologia della freccia a 2 punte. 
Di fatto gli elettroni che formano il legame doppio sono dislocati in tutta la molecola. 

\begin{itemize}
    \item Ottetti incompleti (dove minore di 8): composti di Be e B
    \item Ottetti espansi (dove maggiore di 8): è caratteristico dei non metalli con $n\ge 3$ che possono accedere anche agli orbitali vuoti, per la differenza non troppo elevata
\end{itemize}

\subsubsection{Radicali}
Tutte le specie con \emph{elettroni spaiati} si dicono \textbf{radicali}. Sono \emph{paramegnetici} ed estremamente reattivi (sono coinvolti nel cosidetto smog fotochimico). 

\subsection{Nota importante sulla teoria di Lewis}
La struttura di Lewis ci dà informazioni solo riguardo il numero di legami che si possono formare e trova riscontro qualitativo con le energie di legame, ma non ci dà alcuna informazione sulla geometria della molecola stessa, si parla infatti di struttura dei legami e non di geometria molecolare. 

\section{Geometria molecolare}
I legami covalenti hanno una precisa lunghezza e una certa direzionalità, va quindi a determinare la disposizione spaziale degli atomi. Una buona teoria del legame deve considerare distanze \emph{d} e angoli \emph{$\alpha$}. Di fatto si unifica la teoria di Lewis con la \textbf{teoria di VSEPR} per darci una geometria molecolare approsimativamente corretta. 

\textbf{VSEPR}: Valence Shell Electron Pair Repulsion.  

Cerco di minimizzare le interazioni fra legami delle coppie del guscio di valenza. Loro si respingeranno e si disporranno nello spazio il piu lontano possibile. Per esempio in \ce{BeF2} abbiamo due atomi di floro, i quali saranno alla massima distanza indipendentemente dalle coppie solitarie. 

\subsection{Modello elettrostatico}
Le \textbf{cariche} sono libere di muoversi su una sfera e sono vincolate al centro della sfera, e si dispongono per avere la minima repulsione. 

%%disegni delle varie geometrie per numero di legami


Bisogna notare come la geometria della molecola è diversa dalla geometria di legame. Le CS hanno maggiore densità elettronica delle CL e quindi "occupano" più spazio. Ci sono infatti diverse geometrie in base ai vari legami CL e CS. 


\subsection{Legami multipli}
Sono equivalenti a un legame singolo, ai fini della geometria molecolare
\subsubsection{Polarità delle molecole}
Un legame che unisce due atomi di differente elettronegatività non è perfettamente simmetrico gli $e^{-}$ passano più tempo nei pressi dell'atomo con maggiore elettronegatività. 
\textbf{Legame polare covalente}:per polare si intende che esiste un bipolo elettrico in ogni legame, sono quindi dei legami polarizzati, e i due "termini" del legami assumono una carica parziale, positiva o negativa. Si forma quindi un \textbf{dipolo} a cui è associato un \textbf{momento dipolare($\mu$}, ovver il prodotto tra le cariche e la distanza, misurata in Debye. 

La polarità complessiva è data da $\mu_{tot}=\sum \mu$ se $\mu=0$ è non polare, altrimenti è polare. Nota la geometria di una molecola si può stabilire qualitativamente la sua polarità data dalla somma vettoriale dei momenti dipolari di tutti i legami. 


\subsubsection{Energia di legame (energia di dissociazione)}
E' l'energia espressa in $kJ/mol$ del legame covalente, essa aumenta a parità di atomi coinvolti, con l'ordine di legame. 