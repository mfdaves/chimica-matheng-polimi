\chapter{Lezione 4 - Configurazione elettroniche}

\section{Limiti del modello di Bohr}
Il difetto era per l'appunto considerare l'atomo come un modello planetario, dove il nucleo è molto pesante al centro e gli elettroni gli orbitano intorno. 

\section{Principio di indeterminazione di Heisenberg}
Non è possibile avere una prevsione, non si può determinare contemporaneamente la posizione (x) e quantità di moto (mv) di particelle piccole come l'elettrone. Questo è dato soprattutto dalla sua massa molto piccola, di fatto possiamo avere una certezza maggiore di posizione e velocità del nucleo poichè N volte più pesante. \[\Delta x \Delta (mv)\ge \frac{h}{4\pi}\] Questa è chiamata \textbf{relazione di indeterminazione}. Per m costante e nota, : $\Delta x \Delta v \ge \frac{h}{4\pi m}$.
Per vedere un oggetto occorre illuminarlo con una radazione di $\lambda$ confrontabile o inferiori alle dimensioni dell'oggetto.

Nel mondo \emph{macroscopico} la $\lambda$ della luce visibile è sempre al di sotto delle dimensioni degli oggetti osservati i quali risultano nitidi e ben risoluti. 
Nel mondo \emph{microscopico}, invece, la luce può avere energia non trascurabile, rispetto a quella di una particella microscopica, e questo potrebbe modificare in modo, anche significativo, l'energia di quella osservata, avviene un \emph{perturbamento di stato} (per questo motivo si dice in fisica quantistica che se non osserviamo le particelle si comporteranno in maniera diversa). 

Esempio: elettrone: $m_e=9,11\times10^{-28}g$ applicando la disequazione ho una minima incertezza sulla posizione : $\Delta x=\frac{h}{4\pi m_e}\approx 10^{-4}$.

Occorre descrivere l'atomo con il modello quantomeccanico.
Nota: l'energia di $e^{-}$ si calcola facilmente utilizzando lo spettro di emissione ma si ha x indeterminato. 

\section{Dualismo onda/particella}
\subsection{Ipotesi di de Broglie (1924)}
Le piccole particelle possono essere anche un po' onde. La materia ha quindi proprietà ondulatorie? 
\[
\begin{cases}
    \Delta E = mc^{2}\\
    \Delta E = h\nu = h\frac{c}{\lambda} 
\end{cases}
\]

Che quindi porta a : \[mc^{2}=h\frac{c}{\lambda}\] e quindi a: \[\lambda=\frac{h}{mc}\] Ricavo quindi che un corpo a una certa velocità v può essere associato a una certa lunghezza d'onda. 

\subsubsection{Dimostrazione (1927)}
Questa teoria viene dimostrata sperimentalmente: 

\begin{itemize}
    \item Masse grandi: discreto, particellare
    \item Masse piccole; continuo, onde
\end{itemize}

(Si parla quindi metodo scientifico, nel senso per cui da una teoria, si passa a una sua modellazione data dallla sperimentazione e successivamente ad una riscrittura e via dicendo.)

\subsection{L'equazione di Schrodinger (1926)}
Ricavo un'onda che descrive gli elettroni come onde materiali stazionarie tridimensionali: 


%%andrebbe riscritto la formula dell'equazione, giusto per informazione. 
x,y,z=coordinate cartesiane \\
$\psi$ = funzione d'onda, ampiezza d'onda\\
E = energia totale, V=energia potenziale\\
La soluzione dell'equazione è appunto $\psi$, quindi $e^{-}$ non è più un oggetto solido di cui possiamo seguire e predirre i movimenti. Si parla quindi di \emph{quantomeccanica}. 

\textbf{Vibrazioni}: onde stazionarie materiali tridimensionali. 

%%questa parte era da rividere con le slide. 

\subsubsection{Interpretazione di Schrodinger} 
L'elettrone è assimilabile ad un'onda materiale stazionaria ($\psi$) nelle 3 direzioni disegnando figure. 

%%copiare slide 14

Si noti che $\psi_{n,l,m_l}$, dove n riprende esattamente lo stesso concetto di n di Bohr. 

\subsubsection{Interpreazione di Born}
$\psi$ non ha valore fisico. Per un onda luminosa $\psi$ rappresenta l'ampiezza e il quadrato dell'ampiezza rappresenta l'intensità della radiazione. $\psi^{2}$ è definita come \textbf{densità di probabilità}, quindi si passa dalla concezione di orbite a orbitali, a quindi nuvole elettroniche, $\psi^{2}$ determina la forma dell'orbitale, una superficie limite, all'interno di tali volumi l'elettrone può essere presente al suo interno al 95\%. E' quindi una rappresentazione grafica di dove, in quale zona dello spazio a sua disposizione, si trovi l'elettrone, è una \emph{informazione statistica}. 

Ricordando che invece le energie si possono dedurre utilizzando gli spettri atomici. 

\section{I numeri quantici}
\begin{itemize}
    \item \textbf{n} = numero quantico principale, numero intero positivo. Determina l'energia dell'orbitale e le sue dimensioni, se n cresce aumenta $E_n$ e $e^{-}$ è mediamente più lontano. Gli orbitali di pari n appartengono allo stesso livello/guscio. 
    \item \textbf{l} = numero quantico secondario. Determina la forma dell'orbitale. Simbologia: s, p, d, f. A ciascuno di essi corrisponde il valore rispettivo da 0 a 3, se $l\ge3$ si ha l'ordine alfabetico. l è definito in un intervallo che dipende da n: $0\le 1 \le n-1$.
    \item \textbf{$m_l$} = numero quantico magnetico. Determina l'orientazione, ci sono 2l+1 orientazioni possibili. Di fatto i valori di $m_l$ dipendono da l, a varia nell'intervallo: $1-l \le m_l \le 1+l$. 
\end{itemize} 

Se n è uguale a 0, vuol dire che ci troviamo sul nucleo, il che è impossibile. 

\section{Forme orbitali}
\subsection{Forme orbitali s}
Si hanno quando l = 0, sono orbitali \emph{sferici}, a singola orientazione, in cui si racchiude il 95\% di probabilità. 

%%inserire i disegni di orbitali e cose simili

Osservando il grafico (r,P), si noti che P ha un massimo in per un particolare raggio: $52,9pm=a_o$, proprio il raggio che aveva rilevato Bohr. Se ho n=2 e $l=0=m_l$, avrò delle zone in cui P=0, poichè avrò n sfere concentriche, e dall'andamento della probabilità si evince che a certi \emph{r} si ha probabilità nulla. Quindi come si spiega il loro passaggio a livelli differente se la loro 
probabilità è 0 di essere in quel punto? E' detto \textbf{effetto tunnel}. 

Si evidenzia un trend generale: 
\begin{itemize}
    \item i massimi relativi aumentano con l'aumentare di n 
    \item la densità elettronica diventa sempre più dispera e meno concentrata
    \item nodi (dove $\psi=0$) all'aumentare di n. 
\end{itemize}

\subsection{Forma degli orbitali p}
Ponendo quindi l=1, $m_l$=-1,0,+1. 

con n=2, e n=3,.., avrò diverse configurazione, dette \textbf{doppio lobo} o \textbf{bilobate}.  

%%aggiungere disegni, vedere come farli

\subsection{Forme orbitali d}
Ponendo l=2, e $m_l$=-2,-1,0,1,2. 
Hanno forma cosidetta \textbf{tetralobata}, assomigliano a quattro uova e posso averla da n = 3. Si ottengono 5 orbitali degeneri che corrispondono a un orbitale sferico. 

%%aggiungere disegni. 

\subsection{Forma degli orbitali f}
Ponendo l=3, quindi da n=4, e avrò quindi sette orientazioni possibili, molto difficile da visualizzare e disegnare graficamente, come gli orbitali successivi, per il nostro corso è necessario avere presente le prime quattro forme di orbitali.

\section{Modello di Bohr vs Teoria Ondulatoria}
Si contrappone l'idea di \emph{orbita} all'idea di \emph{orbitale}, ciò significa che stiamo contrapponendo l'idea di un binario sul quale l'elettrone è predisposto a muoversi con il 100\% di P, al raggio di Bohr ($a_o=5,3pm$), all'idea di \emph{nuvola elettronica}, quindi volumi, delimitati da superfici, chiamate orbitali, nei quali si è stimato l'elettrone possa trovarsi con una probabilità del 95\%, e con massima probabilità al raggio di Bohr (massima non significa del 100\%), di fatto l'elettrone potrebbe essere ovunque,

\section{Energia negli orbitali}
\begin{itemize}
    \item Per $Z=1$ (H), e per tutti gli altri \emph{idrogenoidi}, l'energia dipende solo da n. 
    \item per $Z>1$, atomi polielettronici, l'energia dipende da l, con $En\le E_{np} \le E_{nd}$.
\end{itemize}

\subsubsection{Effetto schermo}
Gli elettroni negli orbitali più interni sono più penetranti, difatto schermano quelli più esterni. Quindi un elettrone più esterno risente non solo della carica nucleare, ma anche degli elettroni che sono molto vicini al nucleo; si parla quindi di schermatura da parte degli elettroni: \[Z_{eff}=Z-\sigma\]

%%vedere slide 28

All'aumentare di Z, l'energia degli orbitali decresce. %%aggiungere grafico
L'aumento dipende dalla capacità penetranti relative e la presenza di elettroni nei livelli sottostanti. 
Si ha un criterio generale detto \textbf{"aufbau"} per prevedere con buona approssimazione l'ordine di riempimento degli orbitali. 




