\chapter{Lezione 2 - Stechiometria e nomenclatura}

%riscivere tutto correttamente, utilizzando la scrittura .tex corretta




\section{Massa molecolare e massa formula}
Gli elementi possono essere costituiti da singoli atomi o da molecole di atomi uguali. I composti invece sono formati da elementi diversi, se il legame è \textbf{covalente} si parla di \textbf{composti} \textbf{molecolari} o \textbf{covalenti} se il legame è \textbf{ionico} si parla di \textbf{composti ionici}. 
Le masse sono misurate in \textbf{uma}: 
\begin{itemize}
    \item Atomi =  massa atomica (uma)
    \item Molecole =  massa molecolare  (uma)
    \item Composti ionici = massa formula  (uma) = unità minima
\end{itemize}

Nei composti molecolari o nei composti covalenti si calcola la massa sommando le masse atomiche degli elementi costituenti. 

Si utilizza però un sistema differente per legare ciò che si vede dal macroscopio a ciò che si vede al microscopico. 
Questo legame esiste e può essere misurato se ho un N sufficiente grande, dove N è uguale al numero di atomi, ho quindi una massa misurabile fisicamente. Posso scegliere un N "comodo" che possa correlarmi numericamente la \textit{uma} e la massa (in grammi). 

\[N_A= 6,022 \times 10^{23}\]è il numero di particelle che devo considerare per avere la conversione da uma a g. Se infatti prendo Na atomi di H peseranno 1 grammo

Il \textbf{numero di Avogadro} permette quindi la conversione dalla massa atomica alla mssa in grammi, grazie a questo numero infatti è nata la \textbf{mole} che viene definita: quantità di materia che contiene esattamente $ 6,022 \times 10^{23}$ unità elementari (atomi).

Na ha unitò di misura $mol^{-1}$. 

\textbf{Massa Molare} (MM):quanto massa in grammi ha la mole di questa sostanza.

Per esempio: considerando $1 mol(C) = 12g = 6,022*10^{23}$ atomi di C. Quindi per $MM(C)=12g/mol$. Per una mole quindi si intende avere $6,022*10{23}$ oggetti, la mole è solo un multiplo. 

\section{Stechiometria}

La \textbf{stechiometria} misura dei rapporti quantitativi (ponderali) delle sostanze che prendono parte alle reazioni chimiche. Devo avere un \textbf{modello chimico}, che traduca le reazioni e mostri la conversione dei reagenti a prodotti, che funzioni: si introducono, quindi, i \textbf{coefficienti stechiometrici} (l'equazione deve essere bilanciata).

Possiamo prendere come esempio: \[\ce{CH4 + 2O2 -> CO2 + 2H2O}\] 

Ovviamente, a questo livello di comprensione del problema non siamo in grado di capire le motivazioni per cui quei reagenti danno quel determinato prodotto, però possiamo dedurne, per via delle leggi ponderali, che il numero di atomi, e quindi le masse, dei reagenti, devono essere uguali e identici, agli atomi, e quindi alle masse, dei prodotti. 

Se andiamo a calcolare il numero di grammi, passando per le moli, possiamo giungere al fatto che a entrambi i membri corrisponderanno 80 grammi; è da notarsi che il numero di moli può essere differnte, dipende ovviamente dalla loro conversione, e quindi dai prodotti

La chimica opera su tre livelli: 
\begin{itemize}
\item macroscopico (osservabile)
\item microscopico (particellare)
\item simbolico (modellstica, equazioni)
\end{itemize}


\subsection{Reagente limitante}
Quando un reagente di \emph{N} reagenti è in quantità limitate rispetto agli altri in moli, quando limita la massima resa (= quantità di prodotto ottenibile dalla reazione).

\subsubsection{Come si determina?}
\begin{enumerate}
    \item Calcolare le quantià di reagenti in moli,
    \item dividere ciascuna quantià per il corrispondente coefficiente stechiometrico,
    \item se i rapporti sono diversi il reagente che ha il rapporto minore.
\end{enumerate}

\subsubsection{Resa percentuale}
La stechiometria fornisce la quantità di prodotto ideale (teorica) ma ci sono sempre delle deviazioni che limitano la resa. \[\text{Resa percentuale} = \frac{\text{prodotto ottenuto}}{\text{resa teorica}}\times 100\]


Se la Resa è uguale a 100 è detta reazione quantitativa.
La resa può anche essere limitata da reazioni collaterali che partono da sottoprodotti. 

Per esempio: \ce{A + B -> C + D} dove D può essere un collaterale della reazione. 

\section{Nomenclatura chimica}
Gli elementi a sinistra e in basso sono \textbf{metalli} a destra invece sono chiamati \textbf{non metalli}, quelli sono detti invece \textbf{metalloidi} oppure \textbf{semimetalli}.

\subsection{Numero di ossidazione}
Detto anche \textbf{stato di ossidazione} indica il numero di elettroni che un atomo acquista/cede/usa nel combinarsi (quindi nelle reazioni). 

N.B. : nelle reazioni solo gli elettroni possono essere scambiati. 


Esistono \text{regole} da rispettare in ordine di importanza che ci permettono di assegnare il numero di ossidazione: 
\begin{enumerate}
    \item il N.O. di un elemento libero è 0. Per esempio: l'elio (\ce{He}) ha N.O. = 0. 
    \item La somma dei N.O. di tutti gli atomi:
            \begin{itemize}
            \item in specie neutre è 0 (\ce{MgCl} essendo neutro, la sommatoria dei numeri di ossidazione di entrambi gli atomi è 0)
            \item in ioni è identico alla carica (\ce{Fe^3+} ha quindi 3 come numero di ossidazione)
        \end{itemize}
    \item I metalli del gruppo 1 hanno sempre come numero di ossidazione +1, mentre quelli del gruppo 2 hanno sempre come numero di ossidazione +2.
    \item Il fluoro (\ce{F}) ha generalmente $\text{N.O.}=-1$.
    \item L'idrogeno (\ce{H}) ha sempre $\text{N.O.}=+1$ eccetto quando è legato ad un metallo $\text{N.O.}=-1$
    \item L'ossigeno (\ce{O}) ha sempre $\text{N.O.}=-2$, fanno eccezione alcuni composti con legami \ce{O - F} ($\text{N.O.}=+2$,) e i perossidi ($\text{N.O.}=-1$,)
    \item Gli elementi dei gruppi 17, 16 e 15 hanno sempre $\text{N.O.}=-1,-2,-3$ nei loro composti binari con metalli
    
\end{enumerate}
\subsection{Numenclatura chimica}
La nomenclatura chimica è di tipo sistematico per classi di composti.

\subsubsection{Cationi metallici}
Derivano da un  metallo per perdita di elettroni. 
Alcuni metalli formano solo un tipo di ione e non occorre specificare la carica, in altre va definita. 

Per esempio : 
\begin{itemize}
    \item \ce{Li^+} | ione litio .
    \item \ce{Ba^{2+}} | ione bario.
    \item \ce{Cu^{+}} | ione rame (I) detto anche ione rame\textbf{oso}.
    \item \ce{Cu^{+}} | ione rame (II) detto anche ione rame\textbf{ico}.
\end{itemize}
La nomenclatura "ione rame (I)" è relativa alla \emph{Notazione di Stock} mentre quella che fa riferimento a "ione rameoso/ico" alla \emph{nomenclatura tradizionale}.

%%forse meglio fare una tabella, molto piu chiaro

\subsubsection{Ossidi metallici}
Gli ossidi metallici derivano da un metallo e ossigeno. 
Anione ossido: \ce{O^{2-}}.
Per esempio: 
\begin{itemize}
    \item \ce{4K + O2 -> 2K2O} detto anche ossido di potassio
    \item \ce{2Mg + O2 -> 2MgO} detto anche ossido di magnesio
\end{itemize}
Però:
\begin{itemize}
    \item \ce{2Fe + O2 -> 2FeO} ci si può riferire sia come \emph{ossido di ferro (II)} oppure come \emph{ossido ferroso}.
    \item \ce{4Fe + 3O2 -> 2Fe2O3} come \emph{ossido di ferro (III)} oppure come \emph{ossido ferrico}.
\end{itemize}


\subsubsection{Idrossidi metallici}
Derivano da un ossido metallico e acqua.
Anione idrossido: \ce{OH^{2-}}. 

Per esempio: 
\begin{itemize}
    \item \ce{K2O + H2O -> 2KOH} detto anche idrossido di potassio.
    \item \ce{FeO + H2O -> Fe(OH)2} detto anche idrossido di ferro (II) oppure idrossido ferroso.
    \item \ce{Fe2O3 + 3H2O -> 2Fe(OH)3} detto anche idrossido di ferro (III) oppure idrossido ferrioco.
\end{itemize}

\subsubsection{Idruri}
Quelli \textbf{metallici}:derivano da un metallo e idrogeno (con carattere ionico). 
Per esempio : \ce{NaH} è idruro di sodio e \ce{CaH2} è idruro di calcio.

Quelli derivanti da \textbf{non metalli} sono classificabili in due tipi: \emph{molecolari} e \emph{idracidi}.
I primi hanno nomi d'uso che non corrispondono alla nomenclatura tradizionale: 
\begin{itemize}
    \item \ce{H2O}: acqua.
    \item \ce{CH4}: metano.
    \item \ce{NH3}: ammoniaca.
    \item \ce{PH3}: fosfina.
\end{itemize}

I secondi, invece, hanno una nomenclatura sistematica: 
\begin{itemize}
    \item Se puri e anidridi si considerano composti molecolari: anione -uro.
    \item Se in soluzione acquosa: -idrico.
\end{itemize}

Ex: \ce{HCl} cloruro di idrogeno, \ce{HF} floururo di idrogeno, in \ce{H2O}: acido cloridico oppure acido fluoridico.
Un caso da notare, particolare, è quello del gruppo \ce{CN-} (cianuro) che si comporta in modo simile: \ce{HCN} cianuro idrogeno e in \ce{H2O} è acido cianidrico.


\subsubsection{Anidridi}
Derivano da un non metallo e ossigeno. 

\begin{itemize}
    \item \ce{CO2} anidride carbon\emph{ica} o diossido di carbonio.
    \item \ce{SO2} anidride solfor\emph{osa} o diossido di zolfo.
    \item \ce{SO3} anidride solfori\emph{ico} o triossido di zolfo.
\end{itemize}

Per esempio il cloro (\ce{Cl}) ha 4 possibili combinazioni, quindi 4 stati di ossidazione: 
\begin{itemize}
    \item \ce{Cl2O} anidride \emph{ipo}clor\emph{osa} | ossido di dicloro
    \item \ce{Cl2O3} anidride clor\emph{osa} | triossido di dicloro
    \item \ce{Cl2O5} anidride clor\emph{ica} | pentossido di dicloro
    \item \ce{Cl2O3} anidride \emph{per}clor\emph{ica} | eptossido di dicloro
\end{itemize}

\subsubsection{Ossiacidi}
Derivano da anidride e acqua, sono quindi acidi e il loro suffisso/prefisso(= anidridi) dipende dal N.O. . 

\begin{itemize}
    \item \ce{CO2 + H2O -> H2CO3} acido carbon\emph{ico}.
    \item \ce{SO2 + H2O -> H2SO3} acido solfor\emph{oso}.
    \item \ce{SO3 + H2O -> H2SO4} acido solfor\emph{ico}.
\end{itemize}
E come nel precedente caso delle anidridi, quando esistono 4 stati di ossidazione, si procede nel seguente modo: 
\begin{itemize}
    \item \ce{Cl2O + H2O -> H2Cl2O2  = 2 \underline{\ce{HClO}}} acido \emph{ipo}clor\emph{oso}.
    \item \ce{Cl2O3 + H2O -> H2Cl2O4 = 2 \underline{\ce{HClO2}}} acido clor\emph{oso}.
    \item \ce{Cl2O5 + H2O -> H2Cl2O6 = 2 \underline{\ce{HClO3}}} aicod clor\emph{ico}.
    \item \ce{Cl2O7 + H2O -> H2Cl2O8 = 2 \underline{\ce{HClO4}}} acido \emph{per}clor\emph{ico}.
\end{itemize}


\subsubsection{Ossianioni}
Derivano dalla dissociazione degli ossiacidi, a seconda del N.O. hanno diversi suffissi/prefissi($\neq ossiacidi$)
 -oso | -\textbf{ito}, -ico | -\textbf{ato}

 \begin{itemize}
     \item \ce{H2CO3 -> 2H+ + \underline{\ce{CO3^{2-}}}} ione carbon\emph{ato}.
     \item \ce{H2SO3 -> 2H+ + \underline{\ce{SO3^{2-}}}} ione solf\emph{ito}.
     \item \ce{H2SO4 -> 2H+ + SO4^{2-}} ione solf\emph{ato}
 \end{itemize} 

Analogamente, se si hanno diversi stati di ossidazione si procede con ipo-ito, ito, ato, per-ato. 

\subsubsection{Sali}
Sono composti ionici derivati da un anione e un catione in rapporto tale da garantire l'elettroneutralità. 
Nomenclatura : "anione" di "catione" ma anche "catione" "anione".

\begin{itemize}
    \item \ce{KF} fluoruro di potassio o potassio fluoruro.
    \item \ce{Zn(NO3)2}  nitrato di zinco o zinco nitrato.
    \item \ce{FeCl3} cloruro ferrico o ferro (III) cloruro.
\end{itemize}  

\subsubsection{Sali idrati}
Alcuni composti ionici cristallizzano intrappolando un numero preciso di molecole d'acqua nel reticolo cristallino ed esso viene indicato dopo la formula. 

\begin{itemize}
    \item \ce{CaSO4 + 2H2O} solfato di calcio diidrato.
    \item \ce{CuSO4 + 5H2O} solfato di rame (II) pentaidrato.
    \item \ce{Na3PO4 + 12H2O} fosfato di sodio dodecaidrato. 
\end{itemize}

Fortunatamente, i sali si possono derivare da varie combinazioni.
%%inserire in caso la slide. 

\section{Reazioni di ossido-riduzione}
Quando il numero di ossidazione di un elemento varia da reagenti a prodotti, si è verificato un trasferimento di elettroni: \textbf{ossidazione} e \textbf{riduzione}. 

Dove la \emph{ossidazione} è la perdita di elettroni identificata dal segno + dell'anione, mentre la \emph{riduzione} è l'acquisto di elettroni, identificato dal segno - del catione.
Per esempio: se X cede a Y un elettrone, X viene ossidato e viene definito come agente riducente, mentre Y viene ridotto e viene definito come agente ossidante, quindi X aumenta il suo numero di ossidazione, mentre Y lo aumenta.