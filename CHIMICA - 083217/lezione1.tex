% lezione1.tex
\chapter{Lezione 1 - Cenni fondamentali}




\section{Introduzione}
La \textbf{chimica} è la scienza che studia la struttura la composizione della \textbf{materia} e le trasformazioni che essa subisce. La \textbf{materia} è tutto ciò che ha massa e volume, le sue \textbf{proprietà}, molteplici, le conferiscono la sua esclusività, possono essere di tipo \textbf{estensivo} oppure \textbf{intensivo}, le prime dipendono dalla quantià, mentre le seconde ne sono indipendenti. Le \textbf{trasformazioni} possono essere di natura \textbf{chimica} o \textbf{fisica}, le prime convertono una o più sostanze ... in altre, le seconde invece ne alterano la forma o stato fisico, ma non la composizione. Cenni al \textbf{metodo scientifico} : inventato da Galileo Galilei, si suddivide in due piani: quello sperimentale e quello teorico; la teoria deve essere sempre comprovata da una sperimentazione e quindi ne consegue un metodo di continuo perfezionamento e approvazione da entrambi i piani.


\section{Atomi e molecole}
\subsection{Introduzione alla materia, stati, classificazione}
Gli \textbf{stati} di aggregazione della materia: solido, gas, liquido. Per spiegare le proprietà macroscopiche della materia occorre considerare il livello microscopico.
Ogni campione di materia può essere classificato come una \textbf{miscela} o una \textbf{sostanza pura} (... non esistono sostanze pure in natura, però in maniera ideale le consideriamo in questo modo).\\
Le sostanze pure possono essere atomi di un elemento, molecole di un elemento oppure molecole di un composto, un composto per altro è materia i cui elementi sono legati da rapporti costanti, a differenza di una miscela omogenea, per esempio, sono chiaramenti separabili al contrario degli elementi che sono invece indivisibili.\\
Una miscela eterogenea invece non è un composto ma ne è un arrangiamento e quindi ha una diversa struttura di per esempio un composto che è formato dagli stessi elementi della miscela (eterogenea) -> convertita tramite una trasformazione chimica si può ottenere da essa un composto. \\
\subsection{Leggi ponderali, teorie atomiche}
\begin{enumerate}
    \item Legge della \textbf{conservazione della massa}: in una trasformazione chimica la massa dei reagenti è uguale alla somma delle masse dei prodotti.
    \item Legge della \textbf{proporzioni definite}:il rapporto tra le masse degli elementi che si combinano per formare un dato composto è costante e non dipende dal metodo di ottenimento (ex. l'acqua via terra è identica, rispetto a quella ottenuta da un ruscello).
    \item Legge della \textbf{proporzioni multiple}: Quando un elemento si combina con un altro elemento dando origine a più di un composto, le masse di uno che si combinano con una data massa dell’altro stanno tra loro in un rapporto espresso da numeri interi piccoli.
\end{enumerate}

\subsection{Teoria atomica di Dalton}
Secondo \textbf{Dalton} tutti gli elementi chimici sono costituiti da particelle estremamente piccole e indivisibili chiamate atomi, questi atomi sono differenti per ogni elemento. Le reazioni chimiche avvengono quando gli atomi si riarrangiano, i composti si creano quando due elementi diversi si combinano e perciò un dato composto è sempre costituito dagli stessi atomi nello stesso rapporto, infine secondo la teoria atomica di Dalton gli atomi non possono nè essere creati nè essere distrutti durante le reazioni chimiche (Democrito docet) e non possono trasformarsi in atomi di un altro elemento (non consideriamo circostanze tipo reazioni nucleari o decadimento atomico).

Dalton fu anche il primo a introdurre un sistema di simboli per gli atomi di diversi elementi cercando anche di visualizzarne la struttura (sbagliando miseramente).

Venne presto rimpiazzata dalla simbologia moderna che è tuttora utilizzata e si basa sull'iniziale del nome latino di ogni singolo elemento, nel caso in cui due elementi dovessero avere l'iniziale uguale, si proseguirà con la seconda lettera del nome latino (Berzelius, 1813).

\subsection{Di cosa sono fatti gli atomi?}
E' evidente come il modello di Dalton sia troppo semplice.
\subsubsection{I raggi catodici}
Inserire esprimento dei raggi catodici, con anche disegno
\subsubsection{Esperimento di Thompson}
Inserire esprimento dei raggi catodici, con anche disegno
\subsubsection{Esperimento di Millikan}
Inserire esprimento dei raggi catodici, con anche disegno
\subsubsection{Esperimento di Millikan}
Inserire esprimento dei raggi catodici, con anche disegno
\subsubsection{I costituenti dell'atomo}
L'atomo è elettricamente neutro qundi ha tanti elettroni quanti protoni, le particelle nucleari sono dette nucleoni (protoni e neutroni). Per indicare diversi tipi di atomo e/o nuclei si usa il termine nuclidi. 

\subsubsection{Le dimensioni dell'atomo}
\begin{table}[ht]
\centering
\begin{tabular}{|c|c|c|c|c|}
\hline
\textbf{Particella} & \textbf{Carica relativa} & \textbf{Carica assoluta (C)} & \textbf{Massa relativa} & \textbf{Massa assoluta (kg)} \\
\hline
Protone & $1.602 \times 10^{-19}$ & 1 & $1.673 \times 10^{-27}$ \\
\hline
Neutrone  & 0 & 0 & 1 & $1.675 \times 10^{-27}$ \\
\hline
Elettrone  & -1 & $-1.602 \times 10^{-19}$ & $\approx \frac{1}{1836}$ & $9.109 \times 10^{-31}$ \\
\hline %%correggere tabella, correggere dati
\end{tabular}
\caption{Proprietà delle particelle subatomiche}
\end{table}

L'atomo è una struttura prevalentemente vuota e la materia è concentrata nel nucleo, la massa dell'elettrone è quasi 2000 volte minore della massa del protone e del neutrone, le densità del nucleo è elevatissima.

\subsubsection{Rappresentazione dei nuclidi}
Inserire immagine
\subsubsection{Isotopi}
Inserire immagini

\subsubsection{Massa atomica}
inserire immagini e note
\subsubsection{Ioni e composti ionici}
Gli \textbf{Ioni} sono una specie che deriva da atomi o molecole neutri, per acquisto o perdita di elettroni. (n.b. solo gli elettroni possono esssere scambiati)



%%ricordarsi di inserire tutte le immagini e di mettere a posto il layout

