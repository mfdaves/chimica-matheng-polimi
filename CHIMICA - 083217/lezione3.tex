\chapter{Lezione 3 - Teoria quantistica e teoria atomica}

\section{Limiti dell'atomo di Rutherford}
Secondo le leggi della fisica \emph{classica}, una particella in moto circolare perde energia, di conseguenza gli elettroni che secondo il modello di Rutherford ruotano attorno al nucleo, \textbf{irradiando} e \textbf{emettendo} energia, sono destinati a rallentare il proprio moto e quindi verrebbe risucchiato dal nucleo per differenza di carica. Cosa ne possiamo dedurre? La fisica del \emph{macroscopico} è inutilizzabile per comprenndere e spiegare particelle \emph{microscopiche}. Per questa motivazione nacque la fisica \textbf{quantistica}, fondamentale nelle sue basi della chimica.

\section{La radiazione elettromagnetica}
\textbf{Luce}: propagazione di campi elettromagnetici. Si hanno campo elettrico e campo magnetico tra loro ortogonali e in oscillazione, dipanandosi in linea retta. Le radiazioni possono essere visibili e non. 

\subsection{Parametri caratteristici della luce}
\begin{itemize}
    \item lunghezza d'onda ($\lambda$) è la distanza tra due creste o valli (viene misurata in m o nm)
    \item frequenza ($\nu$): numero di creste che passano per un punto in un sec, misurata in $s^{-1}$, detti anche Herzt (Hz)
    \item velocità dell'onda (c): velocità della luce indipendente dalla radiazione, nel vuoto $3 \times 10^{8}$. Lega $\lambda$ e $\nu$, sappiamo che $c=\lambda \times \nu$, da cui ne conseguono le relazioni inverse. Si noti che questa definizione della velocità della luce è valida nel vuoto.
    \item Ampiezza (A): spostamento sulle ordinate, dipendente dal numero di onde in fase. Non ha 
\end{itemize}
Lo spettro visibile della luce lo si ha all'incirca fra i 400nm-800nm, si consideri che questa porzione dello spettro che è a noi visibile è rapportata come a 2cm su 4km.

\subsection{Natura ondulatoria e natura corpuscolare}
Sono stati considerati inizialmente come due fenomeni distinti. Le onde possono passare per una \emph{fenditura} a differenza delle particelle, e danno origine a fenomeni di \emph{interferenza} e \emph{rifrazione}. 

La \textbf{luce} era considerata una onda e particella tutto ciò che ha massa.


\subsubsection{La vera natura della luce}
Quindi per la fisica classica la luce ha solo naturea ondulatoria, a cui viene associata un flusso continuo di energia.


Ma diversi fenomeni non trovavano spiegazione: 
\begin{itemize}
    \item radiazione del corpo nero
    \item effetto fotoelettrico
    \item spettri atomici
\end{itemize}
Si svilupperà quindi un nuovo modello, la fisica \textbf{quantistica}.

\subsubsection{Corpo nero}
Il corpo nero è un oggetto che assorbe tutta l'energia \emph{incidente} e la riassorbe completamente sottoforma di radiazione termica. Il colore di questa radiazione dipende \emph{solo} dalla sua temperatura. La colorazione varia quindi regolarmente (la stessa gamma di colori delle stelle). 
Non è importante la composizione chimica del corpo. 
Si parla comunque in generale di un \emph{ideale}, i nostri occhi non vedono nulla. 

%%rivedere questo passaggio, inserire immagine. 

La \emph{teoria classica} definisce bene la parte a frequenze alte, ma non ci azzecca proprio per le basse. 

%%sarebbe comodo inserire qui il grafico

Solo per $\lambda$ molto alte i modelli reggevano, ma lo scostamento per $\lambda$ minori diventava assai notevole. La cosidetta "costante ultravioletta" porta al collasso della teoria classica. 

\textbf{Soluzione}: l'"atto di digregazione", introduce la \emph{quantizzazione} dell'energia (1901). Solo con frequenze ed energie si puo avere uno scambio di energia. %%capire se questa cosa ha senso.

\textbf{Quanto}: unità minima di energia da scambiare: 
\[ E=n h \nu\] 

dove n è il numero di quanti che è un numero naturale, maggiore di 0, h è la costante di Plank $6,026\times10^{-34}$ e $\nu$ la frequenza. 
L'energia di ogni corpo è quantizzata.
Quando la quantità di energia associata a un quanto è bassa, i quanti sono molto piccoli e si comportano come un continuo, a frequenze basse i quanti sono granulari, e si entra in una condizone discreta. Infatti nel continuo i grafici classici concordano con Plank. 

\subsubsection{Effetto fotoelettrico (1888) }
E' l'effetto di produzione di corrente elettrica in virtù della luce. 

%%bisogna inserire all'interno del file informazioni specifiche sull'esperimento
Nell'esperimento abbiamo due piastre metallitiche sottovuoto, devo solo illuminare il metallo e con un \emph{amperometro} misuro. 

\textbf{Osservazioni sperimentali}: 
\begin{itemize}
    \item sotto una certa frequenza posso non avere emissioni di elettroni
    \item se aumento l'\emph{intensità} quindi aumento il numero di raggi che mando, aumenta la corrente fotoelettrica, e misuro più elettroni. \underline{Non} è quindi aumentando la frequenza che si aumenta lo scambio di elettroni. Dalle equazioni di Madvev è il contrario.
    \item in realtà anche a basse intensità con $\nu<\nu_0$, si verifica comunque un'emissione.
\end{itemize}

%%inserire i miei bellissimi disegni.

Vengono inviati lo stesso numero di elettroni ma più veloci, l'energia in eccesso dipende dalla energia cinetica. Tutto ciò è in contrasto con Maxwell poichè la luce non è solo di natura ondulatoria. 

Una spiegazione proposta da \textbf{Einstein} (premio nobel 1921) è quella della quantizzazione dell'energia luminosa, ovvero la radiazione incidente è composta da \textbf{fotoni}, ciascuna ha una propria energia $h\nu$. 

L'\emph{energia cinetica} dall'elettrone emesso è esprimibile come: \[E_e=\frac{1}{2}m_ev^{2}=h\nu-\phi\] dove $\phi$: funzione lavoro del metallo,anche detto lavoro di estrazione (di fatto ha l'unità dell'energia). E' comunque da notare che con $E<\phi$ non avviene una emissione di elettroni. Superata la $\nu_0$ si avrà un $E_e$ tanto maggiore tanto sarà maggiore la frequenza (direttamente proporzionali). Inoltre è bene notare come metalli diversi hanno quindi diverse frequenze soglie. 

Mentre le proprietà del corpo nero non dipendono direttamente dalla commposizione chimica dello stesso, l'effetto fotoelettrico ha comportamento sperimentale diverso da metallo a metallo. 

\section{Gli spettri atomici}
La luce emessa da una sorgente calda dà uno spettro luminoso atomico. 

%%inserire immagine lampadine, prisma, 
%%NEWTON

Dall'esperimento quindi effettuato nel '500 da Newton se ne svolgono di differenti: %%inserire il differente esperimento della boccetta di idrogeno

Si ha un tubo con vuoto contenente pochissima quantità di H, esaminando con al stessa disposizione effettuato da Newton, il quale aveva utilizzato luce bianca, si nota che si ha un spettro di emissione con colori visibili differenti, e lo si nota anche al variare del gas. 
Esempio pratico: le lampade al sodio per illuminazione autstradali. 
Ne è quindi conseguito che ogni spettro corrisponde a un diverso atomo, da millenni si utilizzava infatti lo \emph{saggio della fiamma} ovvero un procedimento per cui tramite il colore della fiamma posso identificare di quale elemento si tratta. 

\subsubsection{Equazione di Balmer}
\[\lambda=B(\frac{n^{2}}{n^{2}-2})\] dove B=costante, e n=intero positivo. 
E' una relazione matematica senza alcun tipo di significato fisico. 

\subsubsection{Equazione di Rydberg (1888)}
\[\frac{1}{\lambda}=R_H(\frac{1}{n_1^{2}}-\frac{1}{n_2^{2}})\]
$R_H=1,097\times10^{7} m^{-1}$  $n_1,n_2=\text{numeri interi positivi}$

\section{Il modello atomico di Bohr}
I postulati di Bohr: 
\begin{enumerate}
    \item L'elettrone ruota attorno al nucleo con velocità \emph{v}, su orbite circolari con determinato raggio \emph{r} ed energia \emph{E} (\textbf{stati stazionari}). le orbite sono descritte dal bilancio tra attrazione elettrostatica e forza centrifuga. 
    \item Condizione di \textbf{quantizzazione}: sono permesse solo orbite che soddisfano questa condizione: \[mvr=n\frac{h}{2\pi}=n\bar{h}\]  %% da rivedere come scriverlo 
    Da ciò ne si può derivare le espressioni di E e r: 
    $E_n=-\frac{R_Hhc}{n^{2}}$ $r_n=a_on^{2}$ 
    dove $a_o$ è il valore dell'orbita più piccola, detto anche \emph{raggio di Bohr}. $a_o=53pm$. Quando n=1 è detto anche \textbf{stato fondamentale} da $n\ge2$ si hanno \textbf{stati eccitati} (il raggio dell'orbita è maggiore di $a_o$).
    Interessante è come non siano permessi salti intermedi ma soltanto salti diretti, questo determina la differenza sostanzialmente che rendeva il modello di Rutherford limitato, quindi non avviene più l'avvicinamento verso il nucleo e quindi il suo collasso. 
    Come visto $E_n=-\frac{R_Hhc}{n^{2}}$ ha energia negativa, e lo è sempre, con picco definito in n=1, e con valore specifico di $E_1=-2,18\times10^{-18J}$
    \item Tutti gli assorbimenti e emessioni di energia sono basati sulla quantizzazione, Ogni assorbimento o emissione è fatta da un numero finito di fotoni. 
    \[\Delta E=h\nu\] per ogni specifica $\Delta E$ ho una determinata frequenza. Se non do abbastanza energia non avviene il salto. Il salto è esattamente il corrispondente in termini di energia.
\end{enumerate}

Difatti se : $E_n=-\frac{R_Hhc}{n^{2}}$: 

\[
\begin{cases}
    \Delta E = E_f - E_i = -R_Hhc(\frac{1}{n_f^{2}}-\frac{1}{n_i^{2}}) \\
    \Delta E = h\nu = h\frac{c}{\lambda}
\end{cases}
\]

Si ottiene che : 
\[ h\frac{c}{\lambda}=-R_Hhc(\frac{1}{n_f^{2}}-\frac{1}{n_i^{2}})\]
\[\frac{1}{\lambda}=-R_H(\frac{1}{n_f^{2}}-\frac{1}{n_i^{2}})\] 
proprio l'equazione di \emph{Rydberg}. 

Un errore proprio del modello supposto da Bohr che è rimasto quello ufficiale per circa 15 anni dalla sua teorizzazione è il fatto di considerare come nel modello di \emph{Rutherford}, un sistema planetario, quindi ad ogni frequenza è associata un orbita. 

Il \textbf{modello di Bohr} spiega soltanto i cosidetti atomi idrogenoidi come per esempio \ce{H,H+,Li2+}

Nei casi di Z>1, la formula generale è: 
\[\]