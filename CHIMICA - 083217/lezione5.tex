\chapter{Lezione 5 - Proprietà periodiche e modelli del legame chimico}


\section{L'ordine degli elementi}
Il problema fu quello di trovare un modo per organizzare, di presentare in maniera organica gli elementi. Non solo tramite le masse atomiche, ma anche per proprietà atomiche. \textbf{Difficoltà}: il numero di elementi noti (pochi) e alcuni valore di MA imprecisi e/o errati. 
\textbf{Dimitiri Mendeleev} però \emph{intuì} che gli elementi in qualche modo si ripetono periodicamente (\emph{tavola periodica}), scrisse diverse versioni della tavola e si rese anche conto dei diversi legami con O, e fu anche in grado di produrre analisi predittive che sono state confermate nei decenni successivi. 
\textbf{Moseley} aggiornò la tavola con la scala degli elementi per numeri atomici e non per massa atomica, difatti per 4 coppie di atomi le posizioni sono cosi invertite. La logica moderna comunque segue le logiche di Mendeleev. 

\subsubsection{Terminologia}: 
\begin{itemize}
    \item gruppi (=colonne), vanno da 1 a 18, e accorpano elementi con proprietà simili.
    \item periodi (=righe) corrispondono al numero quantico principale, ultimo strato che deve essere completato. 
\end{itemize}

\subsubsection{Gruppi}
\begin{itemize}
    \item 1 gruppo : metalli alcalini
    \item 2 gruppo : metalli alcalini terrosi
    \item 3-12 gruppo : metalli di transizione
    \item 16 gruppo : calcogeni
    \item 17 gruppo : alogeni
    \item 18 gruppo: gas nobili
    \item dal 57 al 71 numero atomico sono i lantanidi, mentre dal 89 al 103 sono detti attinidi
\end{itemize}

Rispetto alla tavola periodica che vediamo, per motivi di stampe i lantinidi e attinidi sono inseriti in basso e non all'interno del rettangolo stampato. La disposizione degli elementi secondo Z rileva un andamento periodico delle loro proprietà chimico fisiche. 

\subsubsection{Reattività simili in gruppo}
Stessa configurazione elettronica esterna, ovvero stesso $e^{-}$ nel guscio esterno e negli stessi tipi di orbitali. \[\text{n gruppo} = \text{n $e^{-}$ più esterni} = \text{n $e^{-}$ di valenza}\]

Gli elettroni di valenza sono quegli elettroni coinvolti in legami. 

Diverse reazioni sono comuni a classi di atomi, come per esempio: 
\begin{itemize}
    \item Reazioni del potassio \ce{H2O}: tutti i metalli alcalini (gruppo 1) reagiscono rigorosamente con \ce{H2O} producendo H gassoso
    \item Reazioni del cloro con sodio: tutti gli alogeni (gruppo 17) reagiscono a formare composti ionici detti alogenuri. 
\end{itemize}

Molte delle proprietà periodiche sono legate all'andamento di $Z_{eff}$ (vedere lezione 4), dove $\sigma$ è la costante di schermo, ovvero la carica efficace, responsabile delle proprietà. 

%%rileggere slide

$Z_{eff}$ aumenta molto lungo il periodo, mentre nel gruppo varia poco. 

\section{Proprietà periodiche}
\subsubsection{Raggio atomico}
ovvero il raggio utile, la distanza che si ha quando due atomi si legano, anche se a livello teorico, gli atomi sono infiniti. 
    Il suo valore si ottiene da dati sperimentali.
    Il raggio atomico (di legame): 
    \begin{itemize}
        \item Metalli: raggio metallico, cioè la distanza a metà tra due atomi a contatto fra loro nel metallo solido cristallino
        \item Non metalli: raggio covalente, cioè metà fra i nuclei di 2 atomi identici uniti da un legame covalenti. 
    \end{itemize}
Si vede che è una proprietà periodica, sono molto più grandi in basso a sinistra (unità di misura è il pm, picometro). Aumenta lungo il gruppo (n aumenta, e gli elettroni di valenza sono piu lontani), dimunuisce invece lungo il periodo ($Z_{eff}$ aumenta e aumenta quindi l'attrazione verso il nucleo da parte degli elettroni)
%%ci sarebbero i disegni 
\subsubsection{Energia di prima ionizzazione ($IE_1$)}
E' l'energia necessaria per ionizzare un atomo, quindi necessaria alla rimozione del primo elettrone: è pensabile come l'opposto dell'energia di interazione coulumbiana che unisce elettrone e nucleo. E' inversamente proporzionale di $Z_{eff}$ e di $\frac{1}{r_A}$. L'andamento: aumenta molto lungo il periodo, poichè $Z_{eff}$ cresce e $r_A$ diminuisce, mentre diminuisce poco nel gruppo, questo perchè si ha una $Z_{eff}$ simile ma il raggio invece aumenta. 


Si possono considerare anche energie di ionizzazione successive, quest'ultime aumenteranno sempre, poichè aumenta, $Z_{eff}$, questo succede perchè si vanno a staccare elettroni a gusci più interni, perchè è difficile togliere elettroni a qualcosa di già positivo. Ma il salto diventa enorme quando si rimuove da un livello più interno completo, tocco quindi un livello pieno, dopo che tolgo l'ultimo elettrone di valenza, il livello precedente è pieno. 

\subsubsection{Comportamento metallico}
Il carattere metallico segue un andamento approsimativamente opposto a IE: diminuisce lungo il periodo e aumenta lungo il gruppo. Segue anche il carattere \textbf{acido-base} degli ossidi. Più metallici: ossidi basici. Non metalli: ossidi acidi (anidridi).

\subsubsection{Affinità elettronica (EA)}
Energia liberata durante il processo: \ce{X + e^{-} -> X^{-}}. Espressa come variazione di energia, valori negativi crescente. Quanto facilmente atomi creano ioni negativi, dipende da quanto è intensa la carica del nucleo, e da moltissimi altri fatotori. Possiamo dire che: è più negativa lungo il periodo, e meno negativa lungo il gruppo. Si può dire anche che sia proporizionale a $\frac{Z_{eff}}{r_A}$ 

\subsubsection{Elettronegatività (x)}
Si tratta della forza relativa, capacità di un atomo, di una molecola di attrarre a sè gli elettroni che la legano ad altri atomi. E' un parametro che definisce anche il tipo di legame che si forma: \[|x_A-x_B| = K\sqrt{E_{AB}-\frac{E_{AA}+E_{BB}}{2}}\]
X=elettroneg. E=energia di dissociazione di legame, k=una costante; è molto alta se fra atomi molto diversi fra loro, si ha una scala relativa e adimensionale: da 0.7 (Fr) a 4.0 (F), con riferimento arbitrario, se due atomi sono uguali si avrà un legame con nuvola elettronica simmetrica. E' inversamente proporzionale al carattere metallico, e i poco negativi vengono chiamati elettropositivi. Aumenta molto lungo il periodo ($Z_{eff}$ aumenta, $r_A$ diminuisce), diminuisce poco lungo il gruppo. 
Esistono anche diverse definizioni, quella di Mulliken: \[X=K\frac{IE_1+EA}{2}\]
Ha una scala assoluta e con le dimensioni, è comunque un concetto importante ma spiegato sotto una luce diversa.

\subsubsection{Raggio Ionico}
Si tratta di una stima delle dimensioni dello ione in un composto ionico cristallino, gli ioni positivi sono più piccoli, gli ioni negativi sono più grandi. Se vogliamo confrontarli, posso confrontarli se ioni \emph{isoelettronici}, ovvero con configurazione elettronica esterna identica. E' possibile che siano isoelettronici anche con $Z_{A} \neq Z_{B}$, il raggio dipende anche da $Z_{eff}$ che dipende da Z, quindi se non sono isoelettronici  è complicato confrontarli. 



%%teoria del legame chimica